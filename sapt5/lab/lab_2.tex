\documentclass{article}

\usepackage{algorithm}
\usepackage[noend]{algpseudocode}

\begin{document}

\section*{Algoritmi}

\begin{algorithm}
\caption{Algoritmul lui Euclid}\label{euclid}
\begin{algorithmic}[1]
\Procedure{Euclid}{$a,b$}\Comment{C.m.m.d.c al lui a si b}
\State $r\gets a\bmod b$
\While{$r\not=0$}\Comment{Avem raspunsul daca $r=0$}
\State $a\gets b$
\State $b\gets r$
\State $r\gets a\bmod b$
\EndWhile\label{euclidendwhile}
\State \textbf{return} $b$\Comment{C.m.m.d.c este b}
\EndProcedure
\end{algorithmic}
\end{algorithm}

\begin{algorithm}
\caption{Algoritmi elementari de sortare}\label{sortare}
\begin{algorithmic}[1]
\Procedure{insert}{$T[1 .. n]$}
	\For{$i \gets 2$ \textbf{to} $n$}
		\State $x \gets T[i]; j \gets i-1$
		\While{$j > 0 \textbf{ and } x < T[j]$}
			\State $T[j+1] \gets T[j]$
			\State $j \gets j-1$
		\EndWhile
		\State $T[j+1] \gets x$
	\EndFor
\EndProcedure

\Procedure{select}{$T[1 .. n]$}
	\For{$i \gets 1$ \textbf{to} $n-1$}
		\State $minj \gets i; minx \gets T[i]$
		\For{$j \gets i+1$ \textbf{to} $n$}
			\If{$T[j] < minx$}
				\State{$minj \gets i$}
				\State{$minx \gets T[j]$}
			\EndIf
		\EndFor
		\State $T[minj] \gets T[i]$
		\State $T[i] \gets T[minx]$
	\EndFor
\EndProcedure
\end{algorithmic}
\end{algorithm}

\begin{algorithm}
\caption{Algoritm Greedy}\label{greedy}
\begin{algorithmic}[1]
\Procedure{greedy}{$C$}
	\Comment $C$ este mulţimea candidaţilor
	\State $S \gets \emptyset$ \Comment $S$ este mulţimea în care construim soluţia

	\While{\textbf{not} \textit{soluţie}$(S)$ \textbf{ and } $C \neq \emptyset$}
		\State $x \gets$ un element din $C$ care maximizează/minimizează $select(x)$
		\State $C \gets C \setminus \{x\}$
		\If{$ fezabil(S \cup \{x\})$}
			\State $S \gets S \cup \{x\}$
		\EndIf
	\EndWhile

	\If{$\textit{soluţie}(S)$}
		\State \textbf{return} $S$	
	\Else
		\State \textbf{return} "nu există soluţie"
	\EndIf
\EndProcedure
\end{algorithmic}
\end{algorithm}

\begin{algorithm}
\caption{Algoritmul îmulţirii "a la russe"}\label{russe}
\begin{algorithmic}[1]
\Procedure{russe}{$A, B$}
	\State \textbf{arrays} $X, Y$
	\State $X[1] \gets A; Y[1] \gets B$ \Comment{iniţializare}
	\State $i \gets 1$ \Comment se construiesc cele două coloane
	\While{$X[i] > 1$}
		\State $X[i+1] \gets X[i] \textbf{ div } 2$ \Comment \textbf{div} reprezintă împărţirea întreagă
		\State $Y[i+1] \gets Y[i] + Y[i]$
		\State $i \gets i+1$
	\EndWhile
	\Comment adună numerele $Y[i]$ corespunzătoare numerelor $X[i]$ impare
	\State $prod \gets 0$
	\While{$i>0$}
		\If{$X[i]$ este impar}
			\State $prod \gets prod + Y[i]$
			\State $i \gets i - 1$
		\EndIf
	\EndWhile
	\State \textbf{return} $prod$
\EndProcedure
\end{algorithmic}
\end{algorithm}

\begin{algorithm}
\caption{Şirul lui Fibonacci}\label{fibonacci}
\begin{algorithmic}[1]
\Procedure{fib3}{$n$}
	\State $i \gets 1; j \gets 0; k \gets 0; h \gets 1$
	\While{$n > 0$}
		\If{$n$ este impar}
			\State $t \gets jh$
			\State $j \gets ih + jk + t$
			\State $i \gets ik + t$
		\EndIf
		\State $t \gets h^2$
		\State $h \gets 2kh+t$
		\State $k \gets k^2+t$
		\State $n \gets n$ \textbf{div} $2$
	\EndWhile
	\State \textbf{return } $j$
\EndProcedure
\end{algorithmic}
\end{algorithm}

\end{document}
