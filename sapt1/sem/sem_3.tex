\documentclass{article}

\begin{document}

\pagenumbering{gobble}

\begin{tabular}{| r | l | c | }
	\hline \hline
	Nr. crt & Nume & Ocupaţia \\
	\hline \hline
	1. & Popescu & Inginer \\
	\hline
	2. & Ionescu & Profesor \\
	\hline
	3. & Andrei & Şomer \\
	\hline
\end{tabular}

\vspace{.5cm}

\begin{tabular}{| l l c |}
	\hline
	Nr. crt & Nume & Ocupaţia \\
	1. & Popescu & Inginer \\
	2. & Ionescu & Profesor \\
	3. & Andrei & Şomer \\
	\hline
\end{tabular}

\vspace{.5cm}

\begin{tabular}{|p{4cm}|}
	\hline
	Acesta este exemplu de paragraf scris \^\i n cutie. \\
	\hline
\end{tabular}

\vspace{.5cm}

\begin{tabular}{|p{0.25\textwidth}|p{0.65\textwidth}|}
	\hline
	La\c timea de band\u a (bandwidth)&
	L\u a\c timea de band\u a reprezint\u a cantitatea de informa\c tie care poate fi transmis\u a prin intermediul unui anumit canal de comunica\c tie \^intr-un interval prestabilit de timp. Pentru serviciile digitale l\u a\c timea de band\u a se m\u asoar\u a \^in bi\c ti pe secund\u a.\\
	&Throughput - cantitatea de informa\c tie care estetransmis\u a prin intermediul unui anumit canal decomunica\c tie \^intr-un interval prestabilit de timp. \\\hline
	Motor de c\u autare (Search Engine) & Un program care caut\u a \^intr-un anumit set de documente/site-uri acele documente/pagini ce con\c tin cuv\^antul cheie de c\u autare. Termenul este folosit şi pentru a descrie sisteme complexe ce permit căutarea de documente pe Internet precum Google şi Altavista.\\ \hline
\end{tabular}

\vspace{.5cm}

\begin{tabular}{| c | c | c | r | r | }
	\hline \hline
	\multicolumn{2}{|c|}{Denumirea}& Cant. &
	\multicolumn{2}{c|}{Preţ (mii lei)}\\
	\cline{4-5}
	\multicolumn{2}{|c|}{}&& unitar & total \\
	\hline \hline
	\multicolumn{2}{|c|}{Roşii}& 3 Kg & 0,3 & 0,9 \\
	\hline
	Carne & Cal. I & 2 Kg & 4 & 8 \\
	\cline{2-5}
	& Cal. II & 3 Kg & 2,5 & 7,5 \\
	\hline \hline
\end{tabular}

\end{document}
