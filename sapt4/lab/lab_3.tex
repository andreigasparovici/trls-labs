\documentclass{article}
\usepackage{alltt}

\begin{document}

\begin{enumerate}
	\item Algoritmul lui Euclid

	Descrierea în pseudocod a algoritmului lui Euclid este următoarea:

	\fbox{\begin{minipage}{\textwidth}
	\begin{alltt}
		\textbf{int} a, b, d, i, r;\\
		\textbf{read} a, b; \\
		\textbf{if}(a<b) \{ d=a; i=b; \} \textbf{else} \{ d=b;i=a; \}; \\
		r = d \% i; \\
		\textbf{while} (r != 0) \{ d=i;i=r;r=d \% i; \}; \\
		\textbf{write} i;
	\end{alltt}
	\end{minipage}}

	\item Schema lui Horner
	
	Descrierea în pseudocod a schemei lui Horner este următoarea:

	\fbox{\begin{minipage}{\textwidth}
	\begin{alltt}
		\textbf{int} n, a, b, i;\\
		\textbf{read} n, a, b; \\
		\textbf{int} a[0..n], c[0..n-1];\\
		\textbf{for} i=n,0,-1 \textbf{read} a[i];\\
		c[n-1] = b * a[n];\\
		\textbf{for} i=1,n-1 c[n-i-1] = b * c[n-1] + a[n-i];\\
		val := b * c[1] + a[1];\\
		\textbf{write} val;
	\end{alltt}
	\end{minipage}}

	\item Conversia unui număr natural din baza 10 în baza 2.

	Fie $n$ un număr întreg pozitiv. Pentru a determina cifrele reprezentării în baza doi a acestui număr se poate folosi următoarea metodă:

	Se împarte $n$ la 2, iar restul va reprezenta cifra de rang 0. Câtul obţinut la împărţirea anterioară se împarte din nou la 2, iar restul obţinut va reprezenta cifra de ordin 1 ş.a.m.d. Secvenţa de împărţiri continuă până la obţinerea unui cât nul.

	Descrierea în pseudocod a acestui algotim este:

	\fbox{\begin{minipage}{\textwidth}
	\begin{alltt}
		\textbf{int} n, d, c, r;\\
		\textbf{read} n; \\
		d = n; \\
		c = d / 2; \hspace{.5cm} {\normalfont /* câtul împărţirii întregi a lui d la 2*/} \\
		r = d \% 2; \hspace{.5cm} {\normalfont /* restul împărţirii întregi a lui d la 2*/} \\
		\textbf{write} r; \\
		\textbf{while} (c != 0) \{ \\
		\hspace*{.5cm} d = c;\\
		\hspace*{.5cm} c = d / 2; \\
		\hspace*{.5cm} r = d \% 2; \\
		\hspace*{.5cm} \textbf{write} r;\\
		\}
	\end{alltt}
	\end{minipage}}

\end{enumerate}


\end{document}
