\documentclass{article}
\usepackage{amsthm}

\begin{document}

\newtheorem{obs}{Observaţia}[section]

\begin{obs}
	Dacă $X$ este o variabilă aleatoare discretă ce ia valorile $x_1, x_2, x_3, \dots$ cu probabilităţile $p_1, p_2, p_3, \dots$, atunci au loc următoarele.

	\begin{enumerate}
		\item Dacă $I$ este un interval ce nu conţine nici una din valorile posibile ale variabilei aleatoare discrete $X$, atunci
			\begin{equation}
				P(X \in I) = 0.
			\end{equation}

		\item Probabilitatea ca variabila aleatoare $X$ să ia valori într-un interval $I = (a,b]$ este dată de
			\begin{equation}
				P(a < X \leq b) = \sum_{a < x_i \leq b} p_i,
			\end{equation}

		adică este egală cu suma probabilităţilor $p_i$ corespunzătoare valorilor posibile $x_i$ pentru care $a < x_i \leq b$.

	\item Suma tuturor probabilităţilor $p_i$ corespunzătoare valorilor $x_i$ este egală cu $1$, adică
		\begin{equation}
			\sum_i p_i = 1
		\end{equation}

		Motivul este următorul:
		$$
			\sum_{i \geq 1} p_i = \sum_{i \geq 1} P(X = x_i) = P(X \in \{x_1, x_2, x_3, \dots\}) = P(\Omega) = 1.
		$$
	\end{enumerate}

	Dacă $X$ este o variabilă aleatoare discretă, vom spune că funcţia de distribuţie corespunzătoare este o funcţie de distribuţie \textbf{discretă} (sau că $X$ are o distribuţie discretă).
\end{obs}

\end{document}
